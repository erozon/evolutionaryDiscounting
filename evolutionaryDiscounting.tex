\documentclass[titlepage, hidelinks, 12pt]{article}


%for custom page numbering:
\usepackage{fancyhdr}

\usepackage{lipsum}
\usepackage{hyperref}
\usepackage{palatino}
\usepackage{tikz}
\usepackage{chngcntr}
\counterwithin{figure}{section}
%\usepackage{breqn} % useful for breaking equations across multiple lines automatically. Breaks everything.


\usepackage{setspace}
%\usepackage{indentfirst} %tex default is no indent on first paragraph after heading
\usepackage{url}
\usepackage{amsmath, amssymb, amsfonts, amsthm}
\usepackage{float}
\usepackage{subfig}
\usepackage{graphicx}
\usepackage{environ, enumerate}
%\usepackage{mathbbol}
%\DeclareSymbolFontAlphabet{\amsmathbb}{AMSb}
\graphicspath{ {images/} }
\providecommand{\keywords}[1]{\textbf{\textit{Keywords---}} #1} 
\usepackage[format=plain,
            labelfont={bf, it},
            textfont=it]{caption}

\usepackage{lineno}

\usepackage{epigraph}

% \epigraphsize{\small}% Default
\setlength\epigraphwidth{8cm}
\setlength\epigraphrule{0pt}

\usepackage{etoolbox}

\makeatletter
\patchcmd{\epigraph}{\@epitext{#1}}{\itshape\@epitext{#1}}{}{}
\makeatother

% Align and Lineno do not play well together. This should fix it:

\newcommand*\linenomathpatch[1]{%
  \cspreto{#1}{\linenomath}%
  \cspreto{#1*}{\linenomath}%
  \csappto{end#1}{\endlinenomath}%
  \csappto{end#1*}{\endlinenomath}%
}

\linenomathpatch{equation}
\linenomathpatch{gather}
\linenomathpatch{multline}
\linenomathpatch{align}
\linenomathpatch{alignat}
\linenomathpatch{flalign}



%%%%%%%%%
% indentation
%%%%%%%%%

\setlength\parindent{24pt}

\setlength{\voffset}{-1cm}
\setlength{\textwidth}{17cm}
\addtolength{\textheight}{2cm}
\setlength{\footskip}{1cm}
\addtolength{\oddsidemargin}{-2cm}
\addtolength{\evensidemargin}{-2cm}

\widowpenalty10000
\clubpenalty10000

%%%%%%%%%
% page numbering and logo
%%%%%%%%%

\pagestyle{fancy}

\fancyhead[L, C]{}
\fancyfoot[L]{}
\fancyfoot[C]{\thepage}
\fancyfoot[R]{}
\renewcommand{\headrulewidth}{0pt}
\renewcommand{\footrulewidth}{0pt}


%Def, Lemma, Theorem, Corollary environment
\theoremstyle{plain}
\newtheorem{theorem}{Theorem}[section]
\newtheorem{corollary}[theorem]{Corollary}
\newtheorem{lemma}[theorem]{Lemma}
\newtheorem{proposition}[theorem]{Proposition}
\newtheorem{conjecture}[theorem]{Conjecture}
\newtheorem{question}[theorem]{Question}
%\newtheorem*{proof}{Proof}
\theoremstyle{remark}
\newtheorem*{remark}{Remark}
\newtheorem*{example}{Example}
\theoremstyle{definition}
\newtheorem{definition}[theorem]{Definition}

%New commands
\newcommand{\Q}{\mathbb{Q}}
\newcommand{\Z}{\mathbb{Z}}
\newcommand{\N}{\mathbb{N}}
\newcommand{\R}{\mathbb{R}}
\newcommand{\T}{\mathcal{T}}
\newcommand{\E}{\mathbb{E}}
\newcommand{\betahat}{\hat{\beta}}

\newcommand{\varSS}{\frac{\partial \lambda}{ \partial b_{a}}}
\newcommand{\varLL}{\frac{\partial \lambda}{ \partial b_{a+1}}}
\newcommand{\LH}{\mathcal{LH}}




%New math operators
\DeclareMathOperator{\ringchar}{char}
\DeclareMathOperator{\diag}{diag}
\DeclareMathOperator*{\argmax}{argmax} \DeclareMathOperator{\disc}{disc}
\DeclareMathOperator{\MRS}{MRS}
\DeclareMathOperator{\smallersooner}{SS}
\DeclareMathOperator{\largerlater}{LL}
\DeclareMathOperator{\ES}{ES}
%\DeclareMathOperator{\exp}{exp}
\DeclareMathOperator{\hyp}{hyp}
\renewcommand\d[1]{\:\textrm{d}#1}
\newcommand*\diff{\mathop{\!\mathrm{d}}}

\doublespacing
\begin{document}

\section{Introduction}

Time preferences describe an individual's willingness to wait for a delayed reward in the presence of a more immediate alternative. Quantiatively,
time preferences are described by a \textit{discounting function} $\Delta(t)$. A reward of unit magnitude received at delay $t$ maintains a fraction
$\Delta(t)$ of its value. For example, \textit{exponential discounting} describes the class of discounting functions parametrized by $r>0$:
\begin{equation}
    \Delta(t) = e^{-rt}
    \label{eqn:exponential_discounting}
\end{equation}
It is straightforward to prove that if an individual is indifferent between rewards which are equal in expected value, then exponential discounting
describes time preferences for rewards subject to constant risk. We show that time preferences consistent with optimizing a particular measure of
reproductive success are described by
\begin{equation}
    \Delta^*(t) = \lambda^{-t}P(t)
    \label{eqn:evolutionary_discounting}
\end{equation}
where $\lambda$ is a parameter determined by our measure of reproductive success, and $P(t)$ describes the probabiity of surviving until receipt
of the reward. 


\section{Approach}

In this paper, we derive a discounting function from evolutionary princiles. We start from the assumption that time preferences are
determined by the adaptive unconscience: selective pressures, rather than subjective choice, determine patience. Consuming immediately or later on
will have downstream impacts on reproductive success, and on an evolutionary timescale we expect those time preferences consistent with 
optimizing reproductive success to dominate. 

\section{Tools}
To quantitatively describe a tradeoff between consumption of two goods, economist use a measure called the marginal rate of substitution. To 
quantiatively describe reproductive success, ecologists sometimes describe fitness using a population's unit-time growth rate. 

\subsection{Marginal Rate of Substitution}
An individual's contentment with an endowment of goods $(x_1, x_2)$ can be described using a utility function $u(x_1,x_2)$. The function
maps a quantity of goods to a real number. We then interpret that the allotment $(x_1, x_2)$ is preferred to $(y_1, y_2)$ precisely when
$u(x_1, x_2) > u(y_1, y_2)$. Because $u$ gives a quantitative assessment for the values of different baskets of goods, it additionally allows
us to assess the \textit{relative} value of good $1$ compared with good 2. The \textit{marginal rate of substitution} measures the rate at which
one good is traded off for another at small amounts. 
\begin{definition}
    $\MRS(x_1, x_2) = \frac{\partial u}{\partial x_2}\bigg/\frac{\partial u}{\partial x_1 }$
    \label{def:MRS}
\end{definition}

\begin{example}
    Suppose $u(x_1, x_2) = x_1^{\frac{1}{3}}x_2^{\frac{2}{3}}$. Then $\MRS(x_1, x_2) = 2\frac{x_1}{x_2}$. So given the same allotment of goods 1 and 2,
    $MRS(1, 1) = 2$, and good 2 is twice as valuable as good one. On the other hand, if we already have twice as much of good 2 as good 1, then 
    $MRS(1, 2) = 1$, meaning that a marginal increase in either good is equally valuable. 
\end{example}

\subsection{Fitness}
Fitness is a bit of a rabbit hole, and I'd rather not get into all of the minute details. 

\section{Result}
Here goes the result

\section{Conclusion}

And here we conclude.  
\end{document}

